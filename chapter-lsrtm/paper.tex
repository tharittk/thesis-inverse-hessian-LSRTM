\title{Chain of operators: Experiments}
\renewcommand{\thefootnote}{\fnsymbol{footnote}}
\relax\footnotetext{This work was done under the supervision of Dr. Sergey Fomel.}
\author{Tharit Tangkijwanichakul}
\label{ch:chapter-lsrtm}

\maketitle

\section{Numerical Experiment}
\inputdir{../chapter-lsrtm/pre}

\subsection{Chain as a Deconvolution Filter}
We test the idea of of using the chain of operators with Marmousi data \cite[]{versteeg1994}. The shot gather volume is shown in Figure \ref{fig:bshots45} and consists of 45 shots. The migration velocity is shown in Figure \ref{fig:velmig} which is derived from smoothing the stratigraphic slowness. Figure \ref{fig:bmig1} shows reverse-time migarted image using a finite-difference wave propagator. 

\plot{bshots45}{width=0.4\columnwidth}{Shot gathers}
\plot{velmig}{width=0.4\columnwidth}{Migration velocity for prestack migration}
\plot{bmig1}{width=0.4\columnwidth}{RTM Image}


After we obtaint the sencond migrated image, we estimate $\mathbf{W^{-1}}$ and $\mathbf{W_f^{-1}}$ using the chain of operator. 
We run the chain solver for different numbers of iterations. The residual is about 1.9 $\%$, 0.23 $\%$, 0.21 $\%$ after 2, 5, 10 iterations respectively.These weights are shown in Figure \ref{fig:iw,iw5,iw10} and \ref{fig:iwf,iwf5,iwf10}). Notice that the weights in space domain $\mathbf{W^{1}}$does not change much after more iterations, contrasting to weights in frequency domain $\mathbf{W_f^{1}}$

\multiplot{3}{iw,iw5,iw10}{width=0.4\columnwidth}{$\mathbf{W^{-1}}$ 2, 5, 10 iteration of update}

\multiplot{3}{iwf,iwf5,iwf10}{width=0.4\columnwidth}{$\mathbf{W_f^{-1}}$ 2, 5, 10 iteration of update}


Subsequently, we used chain weights to perform poststack deconvolution as prescribed in equation (\ref{lsmig}). The result (Figure \ref{fig:decon2,decon5,decon10})shows an immediate improvement in resolution improvement over the initial RTM image. The deconvolved image with 5 iterations has higher resolution compared to the one run with 2 iterations. However, the image with 5 iterations gives a smoother image. This may be from the the fact that we get smoother freqeuncy weight $\mathbf{Wf^{-1]}}$.

\multiplot{3}{decon2,decon5,decon10}{width=0.4\columnwidth}{Poststack Deconvolution Image 2, 5, 10 iteration of update}

%
%
%

\subsection{Chain as a Preconditioner for Least-square RTM}

and form a preconditioner according to equation (\ref{prec}). 

\plot{cgrad20}{width=0.4\columnwidth}{LSRTM Image without Preconditioner 20 iterations}

\plot{pgrad20}{width=0.4\columnwidth}{LSRTM Image with Preconditioner 20 iterations $\mathbf{W^{-1}}$ $\mathbf{W_f^{-1}}$}


\plot{spec}{width=0.4\columnwidth}{Freqeuncy spectrum: mig1(blue) cg(red) pcg(pink)}
\plot{zm1}{width=0.4\columnwidth}{Zoom-in Comparison1}
\plot{zm2}{width=0.4\columnwidth}{Zoom-in Comparison2}
\plot{zm3}{width=0.4\columnwidth}{Zoom-in Comparison3}
\plot{zm4}{width=0.4\columnwidth}{Zoom-in Comparison4, note the stair like reflectors}
\plot{cgrad100}{width=0.4\columnwidth}{LSRTM Image without Preconditioner after 100 iterations}
\plot{pgrad100}{width=0.4\columnwidth}{LSRTM Image with Preconditioner after 100 iterations}
\plot{wwpgrad100}{width=0.4\columnwidth}{LSRTM Image with Space only Preconditioner after 100 iterations}

\plot{dres100}{width=0.4\columnwidth}{Normalize data misfit dash=without preconditioner solid(green)=with weight in space only solid(purple)=with chain preconditioner}
