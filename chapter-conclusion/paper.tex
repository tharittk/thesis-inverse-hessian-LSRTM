\title{Conclusions}
\author{Tharit Tangkijwanichakul}
\label{ch:chapter-conclusion}

\maketitle

%\section{Summary}


In this work, we propose a novel approach to approximating the inverse Hessian operator in least-squares migration. The method is data-driven and involves solving for weights in the original and Fourier domains that match two subsequent migrated images using a chain of operators. Our synthetic example shows that the preconditioner formed by this approach can be used to accelerate iterative least-square migration. This results in the final image with improved resolution, balanced amplitudes, and a better data fit.



\section{Future work}
In the future, 

\appendix
\section{Code}
Below is the code for the code of the programs that are newly devoloped for this project. In particular, Mchain2dfft.c and chain2dfft.c are the program for solving for weights in space and frequency domain. Mtf2dprec.c and tf2dprec.c are the subroutine that is used to implement a preconditioning in a Conjugate-gradient solver. The library used in those can be found at Madagascar open-source software environment for reproducible computational experiments \cite[]{madagascar}. The package is available at \url{http://www.ahay.org/}.


\definecolor{mygreen}{rgb}{0,0.5,0}
\definecolor{mygray}{rgb}{0.5,0.5,0.5}
\definecolor{mymauve}{rgb}{0.5,0,0.8}
\definecolor{c0}{rgb}{0.71,0.54,0}
\definecolor{c1}{rgb}{0.75,0.29,0.09}
\definecolor{c2}{rgb}{0.80,0.20,0.18}
\definecolor{c3}{rgb}{0.15,0.54,0.82}
\definecolor{c4}{rgb}{0.52,0.60,0}
\definecolor{c5}{rgb}{0,0.5,0}
\lstset{ 
  backgroundcolor=\color{white},   % choose the background color; you must add \usepackage{color} or \usepackage{xcolor}; should come as last argument
  basicstyle=\scriptsize\ttfamily,        % the size of the fonts that are used for the code
  breakatwhitespace=false,         % sets if automatic breaks should only happen at whitespace
  breaklines=true,                 % sets automatic line breaking
  commentstyle=\color{c5},    % comment style
  extendedchars=true,              % lets you use non-ASCII characters; for 8-bits encodings only, does not work with UTF-8
  frame=single,	                   % adds a frame around the code
  keepspaces=true,                 % keeps spaces in text, useful for keeping indentation of code (possibly needs columns=flexible)
  language=Python,                 % the language of the code
  deletecomment={[s]{'''}{'''}},
  numbers=left,                    % where to put the line-numbers; possible values are (none, left, right)
  numbersep=5pt,                   % how far the line-numbers are from the code
  numberstyle=\tiny\color{mygray}, % the style that is used for the line-numbers
  rulecolor=\color{black},         % if not set, the frame-color may be changed on line-breaks within not-black text (e.g. comments (green here))
  showspaces=false,                % show spaces everywhere adding particular underscores; it overrides 'showstringspaces'
  showstringspaces=false,          % underline spaces within strings only
  showtabs=false,                  % show tabs within strings adding particular underscores
  stepnumber=2,                    % the step between two line-numbers. If it's 1, each line will be numbered
  stringstyle=\color{c2},     % string literal style
  tabsize=2,	                   % sets default tabsize to 2 spaces
  keywords=[1]{from,for,if,in,import,def},
  keywords=[2]{Flow,Fetch},
  keywords=[3]{Result,Plot},
  keywordstyle={\color{c0}},
  keywordstyle=[2]{\color{blue}},
  keywordstyle=[3]{\color{c3}},
}

\lstinputlisting[language=C,label={lst:chain2},caption={chapter-lsrtm/code/Mchain2dfft.c}]{../chapter-lsrtm/code/Mchain2dfft.c}
\lstinputlisting[language=C,label={lst:chain2},caption={chapter-lsrtm/code/chain2dfft.c}]{../chapter-lsrtm/code/chain2dfft.c}
\lstinputlisting[language=C,label={lst:chain2},caption={chapter-lsrtm/code/Mtf2dprec.c}]{../chapter-lsrtm/code/Mtf2dprec.c}
\lstinputlisting[language=C,label={lst:chain2},caption={chapter-lsrtm/code/tf2dprec.c}]{../chapter-lsrtm/code/tf2dprec.c}