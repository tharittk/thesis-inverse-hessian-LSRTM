\author{Tharit Tangkijwanichakul, The University of Texas at Austin}
\title{Abstract}
\label{ch:abs}
I propose a novel way to approximate the inverse Hessian operator by a chain of weights in space and frequency domains. I call this method the Chain of Operators. Fundamentally, we approximate the complex operator (Hessian) via the chain of elementary operators (weights). This method is physically intuitive and provides a simple way to invert for the inverse Hessian. The method can be applied either for compensating migrated images i.e. applying approximated Inverse Hessian directly to the migrated image or in the form of a preconditioner inside iterative least-squares reverse-time migration (LSRTM). 

Tests on synthetic and real data show that this approach provides an effective approximation of the Inverse Hessian while having the minimal cost of forward and inverse FFTs (Fast Fourier Transforms). However, the effectiveness of the methods varies quite greatly from dataset to dataset.

When used for compensating migrated images, the method proves noticeably effective in the synthetic Marmousi dataset. With a real Viking Graben dataset, the method has difficulties in estimating weights due to muted area of the data. However, with some adjustments, the method noticeably improves the image resolution and amplitude balance. Plus, when examining the frequency spectrum, compensating the image by an approximated inverse Hessian emulates the effect of iterative LSRTM. 


When used the approximated Inverse Hessian form a preconditioner in LSRTM, the result with the Marmousi dataset shows that it can significantly accelerate the convergence of LSRTM and achieves high-quality imaging results in fewer iterations. However, when applied to the Sigsbee model, the method is only marginally effective. Particularly. it appears to have difficulties in solving for the weight in the frequency domain as migrating the Sigsbee data with RTM tends to produce stronger low-frequency noise.

Overall, my experiments can be considered as a proof of concept that we can approximate the complex operator via the chain of elementary operators in the particular application to improving the efficiency of Least-squares Migration.